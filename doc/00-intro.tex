\section{Introduction}
\label{s:intro}

\subsection{What is the \rlsim package?}
\label{ss:abstract}

The \rlsim package is a collection of tools for simulating RNA-seq library construction, aiming to reproduce the most important factors which are known to introduce significant biases in the currently used protocols: hexamer priming \cite{hansen10}, PCR amplification \cite{dohm08,benjamini12} and size selection \cite{lee11}. It allows for a systematic exploration of the effects of the individual biasing factors and their interactions on downstream applications by simulating data under a variety of parameter sets. 

The implicit simulation model implemented in the main tool ({\tt rlsim}) is inspired by the actual library preparation protocols and it is more general than the models used by the bias correction methods (e.g. \cite{li10}, \cite{jones12}), hence it allows for a fair assessment of their performance.

Although the simulation model was kept as simple as possible in order to aid usability, it still has too many parameters to be inferred from data produced by standard RNA-seq experiments. However, simulating datasets with properties similar to specific datasets is often useful. To address this, the package provides a tool ({\tt effest}) implementing simple approaches for estimating the parameters which can be recovered from standard RNA-seq data (GC-dependent amplification efficiencies, fragment size distribution, relative expression levels).

The key features offered by the package are the following:

\begin{itemize}
\item{Simulation of priming biases loosely based on a nearest-neighbor thermodynamic model.}
\item{Exact simulation of PCR amplification on the level of individual fragments (consistent across expression levels, no approximations).}
\item{Fragment-specific amplification efficiencies determined by GC-content and length.}
\item{Possibility to simulate PCR and sampling pseudo-replicates.}
\item{Simulation of size selection and polyadenylation with flexible target distributions.}
\item{Estimation of GC-dependent amplification efficiencies from real data, relying on assumptions about locality of biases and the mean efficiency of the fragment pool.}
\item{Estimation of relative expression levels.}
\item{Estimation of empirical fragment size distribution, model selection between normal vs. skew normal distributions.}
\item{Able to simulate experiments on the human transcriptome over a wide range of expression levels on a desktop machine.}
\end{itemize}

The \rlsim package does not simulate the exact physical and chemical processes which take place during library construction, as such a simulation model would involve too many hidden parameters with not enough information in the data about their possible values. 

\subsection{Citing the package}
\label{ss:citing}

An associated manuscript is in preparation, meanwhile the package should be cited as:

\begin{itemize}
\item[]{{Botond Sipos, Tim Massingham and Nick Goldman (2013): {\it \texttt{\rlsim} -- a package for simulating RNA-seq library preparation with parameter estimation} [\href{http://bit.ly/rlsim-doc}{http://bit.ly/rlsim-doc}].}}

\end{itemize}

\subsection{Getting more help}
\label{ss:more_help}

The {\tt BioStar} Q\&A forum (\href{http://www.biostars.org}{\tt http://www.biostars.org}) is an excellent place to get additional help. The author of the package will monitor the posts having the \texttt{rlsim} tag.

\subsection{Dependencies and installation}
\label{ss:dep_install}

\subsubsection{Dependencies}
\label{sss:dep}

The package runs on 64-bit GNU/Linux operating systems.
The \rlsim tool is written in \go \cite{golang} and shipped as a statically linked executable for the \texttt{amd64} Linux platforms, hence it has no dependencies.

\warn{The \rlsim tool can be built for other architectures supported by the \go compiler, however only the \texttt{amd64} architecture is supported and the 32-bit binaries might not work properly.}

The parameter estimation tool (\texttt{effest}) and the additional tools are written in \texttt{Python 2.x} and depend on a couple of packages:

\begin{itemize}
\item \texttt{numpy >= 1.6.2} \cite{numpy}
\item \texttt{matplotlib >= 1.1.0} \cite{plt}
\item \texttt{scipy >= 0.10.1} \cite{scipy}
\item \texttt{biopython >= 1.60} \cite{biopython}
\item{A modified version of the \texttt{HTSeq} \cite{htseq} package available from the GitHub repository under\\
\texttt{\href{https://github.com/sbotond/rlsim/tree/master/misc}{https://github.com/sbotond/rlsim/tree/master/misc}}.

\warn{The official releases of the HTSeq package contain a bug causing segmenation fault when parsing certain paired-end datasets. Please use the modified version from the \rlsim repository!}
}
\end{itemize}

These packages (with the exception of the modified HTSeq) are readily installable from the \texttt{Python} Package Index \cite{pypi} using the \texttt{pip} tool by issuing the following command:

\begin{verbatim}
pip install numpy matplotlib scipy biopython
\end{verbatim}

The additional tool \plotCov tools depend on {\tt samtools} \cite{samtools}.
Simulating Illumina sequencing of the fragments can be done by using the {\tt simNGS} package \cite{simngs} (recommended) or any other sequencing simulator.

\subsubsection{Installing the latest release}
\label{sss:release_inst}

The release tarballs can be obtained from the \rlsim download \cite{rlsim_down} page on \texttt{GitHub}:

\begin{verbatim}
wget https://github.com/sbotond/rlsim/blob/master/releases/rlsim-latest_amd64.tar.gz?raw=true t.tgz -O rlsim-latest_amd64.tar.gz
\end{verbatim}

The unpacked release directory contains the following files:

\begin{itemize}
    \item{{\tt bin/} -- directory containing the executables:
        \begin{itemize}
            \item[]{\rlsim -- the main tool simulating library construction}
            \item[]{\effest -- tool for estimating selected parameters from real datasets}
            \item[]{\sel -- tool for sampling expression levels}
            \item[]{\plotRlsim -- tool for plotting the \rlsim report}
            \item[]{\pbPlot -- tool for visualising sequence biases}
            \item[]{\covCmp -- tool for comparing coverage trends across datasets}
            \item[]{\plotCov -- tool for plotting transcript read coverage colored by reference base}
        \end{itemize}
    }
    \item {\tt COPYING} -- GPL v3 licence
    \item {\tt README.md} -- short instructions in markdown format
    \item {\tt rlsim\_manual.pdf} -- package manual
\end{itemize}

The executables under \texttt{bin/} can be installed by copying them to a directory listed in the \texttt{\$PATH} environmental variable.


\subsubsection{Building from source}
\label{sss:build}

Build dependencies:

\begin{itemize}
\item The package is built using \texttt{make} in a standard \texttt{Linux} environment. 
\item Building the \rlsim tool requires the \go compiler which can be installed as described on the projects website \cite{goinstall}. 
\item A standard \LaTeX\ installation with \texttt{pdflatex} has to be present in order to produce the package documentation. 
\item Regenerating the test datasets needs the {\tt bwa} \cite{bwa} and {\tt samtools} \cite{samtools} commands to be installed.
\end{itemize}

The package source can be obtained by cloning the \texttt{GitHub} repository \cite{rlsim_repo} and built by issuing \texttt{make} in the top level directory:

\begin{verbatim}
git clone https://github.com/sbotond/rlsim.git
cd rlsim
make
\end{verbatim}

A tarball can be built by issuing:

\begin{verbatim}
make release
\end{verbatim}

\subsection{Examples}
\label{ss:examples}

The following basic examples can be run from the \texttt{src/} directory in the package source tree:

\begin{itemize}
\item{Re-sample expression levels from a mixture of gamma distributions with two components with mean 5,000 and 10,000:

\begin{verbatim}
$ ../tools/sel -d "0.5:g:(5000, 0.1) + 0.5:g:(10000, 100)" \
test/basic/test_transcripts.fas > my_transcripts.fas
\end{verbatim}

The output {\tt my\_transcripts.fas} is a Fasta file annotated with the expression levels:

\begin{verbatim}
>ENST00000371588$9430
GCTTCCGGCATCTGGCTCAGTTCCGCCATGGCCTCCTTGGAAGTCAGTCGTAGTCCTCGCAGGTCTCGGCGGGAGCTG ...
...
\end{verbatim}

}
\item{Simulate 10,000 fragments with default parameters, plot \rlsim report:

\begin{verbatim}
$ ./rlsim -n 10000 my_transcripts.fas > frags.fas
$ ../tools/plot_rlsim_report
\end{verbatim}

The output file {\tt frags.fas} contains the simulated fragments:

\begin{verbatim}
>Frag_0 ENST00000374005 (Strand - Offset 1883 -- 2298)
AGAGAATAGAGGGTAGAAGGGAAATTCTTGGCACCTGGACTAGAGTGAGATAAAAGGAGAGTAGGAAAGCAGTGA ...
...
\end{verbatim}

}
\item{Simulate paired-end sequencing using {\tt simNGS} \cite{simngs} and the runfile shipped with the package source:

\begin{verbatim}
$ cat frags.fas | simNGS -p paired -o fastq -O reads test/cov/s_4_0066.runfile
\end{verbatim}

The files {\tt reads\_end1.fq} and {\tt reads\_end1.fq} contain the simulated paired-end reads:

\begin{verbatim}
@Frag_24929 refB (Strand - Offset 11583 -- 12067) 151M
GCCCCGAGTAGTTCTGGGCGGGGCCCCCGCGGCCAGCGCCGCCCACTATATATTATTTATTCTAACTATT ...
+
GGGEFG(EGFGGDGBFEGGG;FD7GEGGDGGA=GGGDG7FFGGGGGFGCAGGGGGFF?GGGFG@GGAGGG ...
...
\end{verbatim}

}

\item{Simulate 500,000 fragments, skew normal fragment size distribution with a spike, {\tt after\_prim\_double} fragmentation method, 15 PCR cycles with the specified efficiency parameters, using 4 cores, verbose mode:

\begin{verbatim}
$ ./rlsim -n 50000 -d "0.9:sn:(600,50,4,300,2000) + 0.1:n:(700,1,600,2000)" \
-f after_prim_double -c 15 -eg "(1.0,0.5,0.8)" -el "(0.1,0.7,1.0)" \
-t 4 -v my_transcripts.fas > frags.fas

10:09:14.327740 Starting up rlsim using maximum 4 cores.
10:09:14.327809 Initial random seed: 1352196554
10:09:14.328244 Number of requested fragments: 50000
10:09:14.328250 Target fragment length distribution components:
10:09:14.328275         0.10:n:(700, 1, 600, 2000)
10:09:14.328280         0.90:sn:(600, 50, 4, 300, 2000)
10:09:14.369978 Finished sampling target lengths.
10:09:14.370402 Fragmentation method: "after_prim_double" with parameter 0.
10:09:14.370564 Number of PCR cycles: 15
10:09:14.370593 GC dependent efficiency parameters: (1,0.5,0.8)
10:09:14.370613 Length dependent efficiency parameters: (0.1,0.7,1)
10:09:14.378443 Poly(A) tail distribution components:
10:09:14.378484         1.00:g:(150, 1, 0, 220)
10:09:14.378501 Fragments will be cached to rlsim_gob_5704.
10:09:14.378504 Fragmenting transcripts and amplifying fragments:
10:09:14.381961         0       ENST00000371588$855
10:09:14.393732         1       ENST00000367772$981
10:09:14.425730         2       ENST00000359326$0
10:09:14.425893         3       ENST00000374005$9899
10:09:14.629702         4       ENST00000002165$8381
10:09:14.736127 Initialized 4 transcripts.
10:09:15.474069 Total number of fragments in the pool: 7.29577e+06
10:09:15.474144 Sampling ratio: 0.0068532862192750045
10:09:15.474193 Sampled 50000 fragments.
10:09:15.474211 Missing fragments: 0

$ ../tools/plot_rlsim_report
\end{verbatim}

}
\item{Estimate parameters from SAM file \red{sorted by read name} using \effest (verbose mode):

\begin{verbatim}
$ ../tools/effest -v -f ../tools/test/ref.fas ../tools/test/aln1.sam 

[12-11-06 10:10:19] Assumed mean efficiency: 0.87
[12-11-06 10:10:19] Number of PCR cycles: 11
[12-11-06 10:10:19] Sliding window step ratio: 10
[12-11-06 10:10:19] Minimum mapping quality: 10
[12-11-06 10:10:19] Saving estimated raw parameters to file: raw_params.json
[12-11-06 10:10:19] Parsing fragments from file: ../tools/test/aln1.sam
[12-11-06 10:10:22] Total number of fragments: 47142
[12-11-06 10:10:22] Total number of fragments from single isoform genes: 47142
[12-11-06 10:10:32] Saving fragment counts to file: effest_counts.pk
[12-11-06 10:10:32] Calculating fragment prior
[12-11-06 10:10:41] Saving fragment prior to file: effest_pr.pk
[12-11-06 10:10:41] Estimated minimum efficiency: 0.513163
[12-11-06 10:10:41] Estimated shape parameter: 5.57905
[12-11-06 10:10:43] Saving estimated expression levels to file: _expr.fas
[12-11-06 10:10:43] Estimating fragment size distribution (truncated normal)
[12-11-06 10:10:43] Estimated mean fragment length: 464
[12-11-06 10:10:43] Estimated fragment length standard deviation: 87
[12-11-06 10:10:44] Truncated normal AIC: 553765
[12-11-06 10:10:44] Estimating fragment size distribution (skew normal)
[12-11-06 10:10:44] Estimated location parameter: 364
[12-11-06 10:10:44] Estimated scale parameter: 132
[12-11-06 10:10:44] Estimated shape parameter: 2.7782
[12-11-06 10:10:45] Skew normal AIC: 552143
[12-11-06 10:10:45] Absolute AIC difference: 1621.73

Suggested rlsim parameters:
        -d "1.0:sn:(364, 132, 2.7782, 229, 868)"
        -eg "(5.57905, 0.513163, 1)"
        -n 47142

\end{verbatim}
}

\item{Estimate parameters from SAM file \red{sorted by read name} using \effest -- assume 15 PCR cycles and a pool efficiency of 0.9:
\begin{verbatim}
$ ../tools/effest -v -c 15 -m 0.9 -f ../tools/test/ref.fas ../tools/test/aln1.sam 

[12-11-06 10:12:14] Assumed mean efficiency: 0.9
[12-11-06 10:12:14] Number of PCR cycles: 15
[12-11-06 10:12:14] Sliding window step ratio: 10
[12-11-06 10:12:14] Minimum mapping quality: 10
[12-11-06 10:12:14] Saving estimated raw parameters to file: raw_params.json
[12-11-06 10:12:14] Parsing fragments from file: ../tools/test/aln1.sam
[12-11-06 10:12:17] Total number of fragments: 47142
[12-11-06 10:12:17] Total number of fragments from single isoform genes: 47142
[12-11-06 10:12:26] Saving fragment counts to file: effest_counts.pk
[12-11-06 10:12:26] Calculating fragment prior
[12-11-06 10:12:35] Saving fragment prior to file: effest_pr.pk
[12-11-06 10:12:35] Estimated minimum efficiency: 0.626195
[12-11-06 10:12:35] Estimated shape parameter: 5.61645
[12-11-06 10:12:37] Saving estimated expression levels to file: _expr.fas
[12-11-06 10:12:37] Estimating fragment size distribution (truncated normal)
[12-11-06 10:12:37] Estimated mean fragment length: 464
[12-11-06 10:12:37] Estimated fragment length standard deviation: 87
[12-11-06 10:12:38] Truncated normal AIC: 553765
[12-11-06 10:12:38] Estimating fragment size distribution (skew normal)
[12-11-06 10:12:38] Estimated location parameter: 364
[12-11-06 10:12:38] Estimated scale parameter: 132
[12-11-06 10:12:38] Estimated shape parameter: 2.7782
[12-11-06 10:12:39] Skew normal AIC: 552143
[12-11-06 10:12:39] Absolute AIC difference: 1621.73

Suggested rlsim parameters:
        -d "1.0:sn:(364, 132, 2.7782, 229, 868)"
        -eg "(5.61645, 0.626195, 1)"
        -n 47142

\end{verbatim}
}

\item{Simulate 20,000 fragments using the raw parameters estimated by \effest, set minimum GC-dependent efficiency to 0.5 (verbose mode):

\begin{verbatim}
$ ./rlsim -v -n 20000 -j raw_params.json -jm 0.5 my_transcripts.fas > frags.fas

10:13:50.240213 Starting up rlsim using maximum 4 cores.
10:13:50.240353 Initial random seed: 1352196830
10:13:50.240887 Using raw parameter file: raw_params.json
10:13:50.241033 Number of requested fragments: 20000
10:13:50.314449 Finished sampling target lengths from raw size distribution.
10:13:50.314969 Fragmentation method: "after_prim_double" with parameter 0.
10:13:50.315094 Number of PCR cycles: 15
10:13:50.315114 Using raw GC efficiencies with a minimum efficiency: 0.5
10:13:50.315956 Poly(A) tail distribution components:
10:13:50.316001         1.00:g:(150, 1, 0, 220)
10:13:50.316043 Fragments will be cached to rlsim_gob_5760.
10:13:50.316058 Fragmenting transcripts and amplifying fragments:
10:13:50.317036         0       ENST00000371588$855
10:13:50.328412         1       ENST00000367772$981
10:13:50.359313         2       ENST00000359326$0
10:13:50.359447         3       ENST00000374005$9899
10:13:50.561491         4       ENST00000002165$8381
10:13:50.672723 Initialized 4 transcripts.
10:13:50.987213 Total number of fragments in the pool: 1.28970086e+08
10:13:50.987261 Sampling ratio: 0.00015507472019519317
10:13:50.987331 Sampled 20000 fragments.
10:13:50.987386 Missing fragments: 0

$ ../tools/plot_rlsim_report
\end{verbatim}

}
\end{itemize}

